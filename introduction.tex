%%%% INTRODUCTION
%% \section{Overview} % seems more appropriate than Introduction, at this point
%% \subsection{Background}
%% \subsection{Problem Definition and Proposed Solution}
%% \subsection{Delivery}
%% \subsection{Solution Specifications}
%% \subsection{Deployment Phases}
%% \subsection{Broader Engagement}
222
While cloud computing has been enourmously successful for internet
services, its potential is much greater.  It could support interactive
supercomputing applications that are not possible today. For example,
consider a mathematica like environment~\cite{julia} that spins up
thousands of nodes for a second to allow an engineer to work
interactively with massive problems.  Or, a medical imaging
application that allows a clinician to perform motion correction on
MRI images and interact with them in real time~\cite{ebbrt}.  It could
enable new cyber-physical applicatons, where the rich compute-intesive
applications interact with massive data sets and in real time with the
physical world.  It could provide users with strong and verifyable
security gaurantees, allowing broad adoption by even the most security
sensitive segments of society.  It could, host massive data sets, and
enable computationally intensive Big Data solutions to be used by
anyone.  It could be orders of magnitude less expensive and be an
integral part of a smart power grid.

To achieve this, innovation will be required at all levels, including
new OSes, new hypervisors, new middleware for different domains, new
Big Data platforms, new distributed programing models, new storage technologies,
new networking technologies, new accellerators, and new cloud 
services.  It will require the innovation of a broad community of
researchers and industry, working together, building on each other's
innovation, and competing with each other.

Unfortunately, the current model of cloud computing limits many types
of innovation to a small number of players.  The fundamental problem
is that each cloud is stood up by a single provider who is responsible
for all aspects of its implementation.  While researchers (and
industry at large) can innovate on top of a cloud, only the provider
can innovate in the cloud implementation.  Also, research on top of
the cloud is, in many cases, limited by the underlying capabilities of
the cloud.  For example, one can't explore real time use of a cloud if
the cloud doesn't provide the required services.  Any examination of
the top CISE conferences shows that most publications in innovating in
cloud comes from researchers at a small number of large companies.

We need a very different model of cloud if we are going to enable
innovation by the research community.  A new testbed to enable
research on cloud computing (Section~\ref{requirements}) must expose
researchers to real problems and real workloads, must involve
industry, and must have an architecture that enables researchers to
focus on a problem, while being able to depend on and leverage a
broader infrastructure to deploy and exercise their innovation.

Our vision on how the cloud should evolve is as a marketplace or
exchange, where many entities can simultaneously innovate at many
different levels, and clients can select and compose the services
provided by these different entities.  We call this model an Open
Cloud eXchange (OCX).  Each provider of a service, or eXchange Service
Provider (XSP), is responsible for developing and operating their own
service and exposing sufficient information to allow the customer to
select and interact with their service.  The OCX operator is
responsible for shared networking, billing, data center operations,
and operating the exchange services that allow the customer to select
between XSPs.

The OCX model would offer enourmous value for researchers.  A
researcher can focus on their own service, not having to worry about
all the other services needed to implement a full cloud.  A researcher
can develop a new service by layering it on top of other existing
services.  For example, they can build a new instance of a compute
service by handling a few special cases, and forward all requests
they don't want to handle to other services.  New services that
provide entirely new functionality can be exposed alongside existing
ones.  On the other hand, a researcher that wants to expose, for
example, a new cloud service can install their service on bare
computers, and then make it a part of overall service available to
customers.

This proposal describes a four-year project to develop the OCX Research
Testbed (ORT).  This testbed will be the first instance of an OCX, and
will be designed to empower the research community to actively engage
in research on and in cloud computing.  At the same time, the ORT will
have a strong partnership with industry, providing it: 1)~a model for
long-term self sustainability, 2)~a path for the innovation developed in
the cloud to be visible to and adopted by industry, and 3)~a mechanism
to ensure that the research is informed by the very rapid changes
going on in the commercial cloud sectors.  

Section~\ref{requirements} describes the requirements that a cloud
testbed should meet and the research that it should enable and
Section~\ref{sec:arch} describes the OCX model on which we are basing
the ORT.

Section~\ref{sec:realization} describes our strategy for realizing the
testbed based on this model.  The ORT will be a single distributed
cloud federating capacity at two sites: one in Boston region and one
in Tennessee.  Both of these sites will leverage major NSF, state,
academic and industry investments. Both of these sites are a hub for a
community of universities that are critical to the testbed.  Also, we
will leverage in this effort private clouds and industry, provide us a
path forward to achieving the aggressive goals in the time frame and
budget of this proposal.

Section~\ref{sec:projects} describes specific projects that include
those needed to create an operational testbed, specific technologies
that will be developed or enhanced to expand the role of the testbed,
and domain-specific projects to enable specific research communities.

Section~\ref{sec:inf} describes the initial infrastructure on which
the first phase development will be pursued, and some of the other
infrastructure available at the facilities on which we are standing up
the testbed that we plan to exploit for the project.  The project will
leverage the investments that have been made in a number of clouds and
testbeds as discussed in Section~\ref{sec:testbeds}.

A critical part of developing a testbed is to engage a broad
community.  Our plans for doing so are described in
Section~\ref{sec:com}. We discuss in Section~\ref{sec:req} how the ORT
will meet the challenges discussed above.  Finally,
Section~\ref{sec:pm} discusses how the project will be managed. 


%% \subsection{Requirements for a Testbed}
%% \label{requirements}
%% \input{project_description/challenges}
%% \subsection{Proposal Overview}
%% \input{project_description/overview}


%%  LocalWords:  eXchange OCX XSP XSPs Testbed ORT testbed
